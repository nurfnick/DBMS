\documentclass[11pt]{article}
\usepackage{hyperref}
\usepackage{amsthm}
\usepackage{amsmath}
\usepackage{amsfonts}
\usepackage{tikz}
\usepackage{ wasysym }

\newtheorem{example}{Example}


\author{}
\title{}

\begin{document}
%\maketitle
{\Large
%Change Document name to: Graded Homework 1\_Jacob\_Nicholas
\noindent NAME:  Nicholas Jacob\\ 
STUDENT ID: \# 113578513\\
GRADED HOMEWORK NUMBER: 2\\
COURSE: CS/DSA 4513 DATABASE MANAGEMENT\\ 
SECTION: ONLINE\\SEMESTER: FALL 2023\\
INSTRUCTOR:  DR. LE GRUENWALD\\
 SCORE:}

\newpage
\begin{enumerate}
\item ER and Relational Databases
\begin{enumerate}
\item The ER diagram included in the assignment could be changed into the following story.
\begin{enumerate}
\item A new venture, has a number of venues.  Each venue is identified by its name and address and includes an organizer.  Additionally each venue can be staffed or unstaffed.  A staffed venue will include a budget and an unstaffed venue has a total area.
\item Each venue has a number of different rooms which are identified by number and have a limited capacity.
\item Each staffed venue is controlled by a member.  They have a unique id, a name, and possibly multiple phone numbers.
\item Each staff member was hired for a certain job to work on with a job id and payment established. Additionally these jobs have a start and end date.
\item The staff members all sign a contract.  Contracts include an id, date signed and any additional details.
\item The venues can be reserved by companies.  The reservation agreement will include an id and payment.  The reservation data is also included.
\item Company information is kept including parent company information including any dates on which they were acquired.
\end{enumerate}
\item Relational Database

Venue(\underline {name} \underline{address} organizer)

\end{enumerate}



\end{enumerate}



\end{document}